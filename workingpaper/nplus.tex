\documentclass[english,12pt]{article}
\usepackage[T1]{fontenc}
\usepackage{geometry}
\geometry{verbose,bmargin=2.5cm,lmargin=2.5cm,rmargin=2.5cm}
\usepackage[utf8]{inputenc}
\usepackage{amsmath}
\usepackage{amsfonts}
\usepackage{amssymb}
\usepackage{rotfloat}
\usepackage{wasysym}
\usepackage{graphicx}
\usepackage{natbib}
\usepackage{latexsym}
\usepackage{caption}
\usepackage{flafter}
\usepackage{babel}
\usepackage{imakeidx}
\usepackage{amssymb,amsmath}
\usepackage[table]{xcolor}
\usepackage[mathlines,displaymath]{}
\usepackage{anyfontsize}
\usepackage{verbatim}

\newcommand{\etal}{{et~al.{}}}
\newcommand{\ie}{{i.~e.{}}}
\newcommand{\eg}{{e.~g.{}}}
\newcommand{\viz}{{viz.{}}}
\newcommand{\etc}{{etc.{}}}
\newcommand{\apriori}{{a priori{}}}
\newcommand{\vv}{{vice versa{}}}
\newcommand{\cf}{{}}
\usepackage{titling}
\usepackage{color}


\date{}

\topmargin 0.0cm
\oddsidemargin 0.5cm
\evensidemargin 0.5cm
\textwidth 16cm 
\textheight 22cm

\makeindex
\begin{document}
\begin{flushleft}
\textbf{\Large {$\mathcal{ROBHOOT}$} -- An Open Platform for data integration, inference and prediction}
%by $\mathcal{N}$+3 -- {\small Computing worldwide access to ideas}}
\\
%%\vspace{1.0cm} Alejandro Rozenfeld$^{1}$, Charles N. de Santana$^{1}$, Carlos J. Meli\'an$^{1}$
\\
%\vspace{1.0cm} \bf{1} Horizontal Networks Center. Infinite Galaxy Road, Via Lactea.
\\
\begin{figure}
\vspace{-3 in}
\begin{center}
\includegraphics[scale=0.4]{robhoot.pdf}
\end{center}
\caption{Our dream icon $\mathcal{ROBHOOT}$}
\end{figure}
\end{flushleft}

\newpage


\begin{comment}
\carlos{PUNTOS A SOLUCIONAR\\
  \\
  \\
  1) Comparar codigos Modern Portfolio Theory para generar carteras. Usando como ejemplo el archivo Stocks.mat seleccionar N carteras con 20 stocks cada una. Coinciden las carteras? (Viernes 29 de Enero)
  \\
  2) Primeras carteras sobre Abril Mayo 2016. Incluidas diversificacion geografica y transicion entre carteras para minimizar comisiones
  \\  
  3) Cuenta de saxobank ya creada y el primer deposito de 12000USD hecho
  \\
  4) Lo ideal seria tener un manual avanzado del package ROBHOOT a finales del 2016. A partir de entonces se podria poner en github. Mientras seria ideal ir subiendo todo al server de sharelatex.
  \\
  5) Comparar codigos de julia y octave usando los dos quandl. Porducen el mismo output? Lo ideal seria compararlos todos. 
  \\
  6) Listado de productos de saxobank. Mirar en Quandl si solo con los nombres del pais ya sepuede acceder al listado de todos los productos. Consultar con Adrian de saxobank si existe ya el listado.
\end{comment}

\newpage


\tableofcontents
\newpage

\section{Summary}

High-resolution data coming from many sources is here. Yet, inferring
insightful patterns and drivers remains challenging in many
disciplines. Robhoot aims to develop approaches to integrate data,
Statistical Learning, AI algorithms and process-based models to take
better informed decisions in research, management and investment
landscapes.

\begin{comment}
  Data acces markets are highly regulated and free access to
  high-resolution data and investment models to accurately track
  market trends remains a challenge. Most companies offering services
  to small investors show a trade-off between free individual
  investment decisions at high fee rates and decreasing investment
  cost at the expense of freedom (i.e., automated investments). Here
  we present the package $\mathcal{ROBHOOT}$ in julia to join free
  access to high-resolution data with a suite of investment models in
  one open access platform to investigate markets. Our goal is to
  allow small investors to access and combine fast and free worldwide
  high-resolution data and models to inform free-individual decisions
  and automated algorithms to track highly diversified portfolios.
\end{comment}
\\
Keywords: automated data integration, multilayer networks, approximate Bayesian computation, regulatory markets
\newpage

\begin{comment}
\section{Investigating and investing in highly regulatory markets}

Strong resource dependence is a trade-off. One one side, obtaining all
resources from one labour and employer may help to make us more
specialized and efficient in completing some tasks. On the other side,
it may monopolize the surplus produced by individual's well oriented
ideas, risk and merit. Here we aim to facilitate informed portfolio
diversification for individuals that want to diversify risk and
merit. Any risk and merit fall within regulatory markets. Regulatory
markets control individual decisions to investigate for low cost a
worldwide connected and distributed market. Thus, it is important to
know the meaning of a regulatory market when diversifying
portfolios. In addition, we have to know how new ideas are being
developed in a highly regulatory market to overcome the high cost of
investments for small investors.
\end{comment}

\section{Data Collection (DC)}
We collect and clean data received from different sources. The
collected data can be available in CSV, database, or "real time"
(e.g. [Nakamoto Terminal](https://www.nterminal.com), [BigQuery](
https://cloud.google.com/bigquery/)). We aim to have a package in
Julia language and let the user automatically get the data in their
desired format.

Most data access platforms, from genomes to ecosystems and markets of
any kind are highly regulated. This means most interacting agents in
the market have to deal with a highly complex set of regulations
before having access to the data, algorithms and the available
strategies. Having ``easy'' access to the information in a ``perfectly
informed market'' should be simple and efficient, but unfortunately,
it is not. 


\section{Complexity Reduction (CR)}

To increase performance, we would reduce data complexity with methods
such PCA....etc

\section{Pattern Inference (PI)}

Robhoot will range from classical variance-covariance matrices to AI
algorithms and process-based methods.

\section{Model Validation (MV)}

Model validation can be Bayesian Inference, Approximate Bayesian
Inference, and common goodness of fit methods.



%Take from here to edit main points
\begin{comment}
Open access to the data in a highly regulated market in biology and markets in general\\
Introduce Quandl access to the data in julia: pros and cons\\
\\
The first thing to consider is to have access to the data in a regulatory market. 
In order to get the data, we should know each product ID (e.g., as ID's for
species), the many to one tickers underlying each ID and how these
ID's apply to the products ranging from bonds to stocks\footnote{Link
  showing a full description of the available investment
  products}. There have been attempts to produce an ID following
international standards
\footnote{International Securities Identification Number, ISIN, \url{http://www.isin.org/isin-process/}}
\index{International securities identification number (ISIN)} (Figure
1 and 2). Each ID contains a time series, how and why each ID
fluctuate in a given market? How are markets connected? Which are the
underlying mechanisms producing the fluctuations and the correlations
between products and markets? Are highly correlated markets only
occurring in severe crisis? How regulatory markets influence
individual decisions to have free access to investigate and invest for
low or no cost in a worldwide connected and distributed market? Does
regulation mean higher cost to access the data and invest? Which is
the best (e.g., lowest cost?) strategy to invest in highly regulatory
markets?

The first step to answer these questions is to have easy and universal
access to the data is a first step and there are a few attempts for
such universal ID's. For example, the $Bloomberg$ $Open$ $Symbology$
data since 2012 contains a global and universal ID for each product
and it allows to mappings to exchange tickers, ISINs and other
financial data resources \footnote{Bloomberg open symbology,
  \url{http://datahub.io/dataset/bsym} and
  \url{https://github.com/ga-group/bsym}}. With the open symbology,
obtaining high-resolution and free intra- and inter-day data in an
integrated platform with access to all the markets should be easier
but still the services offering data access and tracking the market
vary greatly and there are no options to have free and open access to
the data with a global ID (\footnote{attempts include jstock
  \url{http://jstock.org/}}. Products range from the more classical
close investment platform \index{close investment platform} to
\index{automated investment platforms} at low
fee\footnote{\url{https://www.sigfig.com/site/#/home/am}}. Close
investment platform describes funds and investment products offered by
a bank or firm a client decides to choose, limiting individual choices
to free tracking and investing and diversifying portfolios.

It is important to distinguish an open investment platform
\index{open investment platform}, on the other hand, where investors
can choose to invest in other funds and vehicles offered by competitor
institutions and companies from open platform in computing \index{open platform in computing}
which refers to a fully documented external application programming
interfaces (API) that allow using the software to function in other
ways than the original programmer intended, without requiring
modification of the source code. Ideally, an open platform in
computing should directly be translated to an open access server where
you can register for free and modify or not the embedded codes to
track your own portfolio, have access to high-resolution data \index{high-resolution data}
and automated investment algorithms \index{automated investments algorithms}.

Most investors are of small size and invest with a low frequency and
high-fee environment \footnote{Have access to the invested-size
  distribution per individual investors}. Thus, returns are small and
can be quickly erased by fees and highly fluctuating markets. For
example, investing in stocks can be very costly if we trade
constantly, especially with a minimum amount of money available to
invest. Every time that we trade stock, either buying or selling, we
will incur a trading fee. Trading fees range from the low end of $10$
per trade, but can be as high as $30$ for some brokers. Remember, a
trade is an order to purchase shares in one company. If we want to
purchase five different stocks at the same time, this is seen as five
separate trades and we will be charged for each one. Now, let's
imagine that we decide to buy the stocks of five companies with an
initial 10,000USD. To do this we will incur from 50USD to 150USD in
trading costs, which is equivalent to 0.5\% to 1.5\% of the initial
10,000USD. If we were to fully invest the 10,000, our account would be
reduced to 9950USD or 9850USD after trading costs. This represents an
important loss, before our investments even have a chance to earn a
cent. In addition, if we were to sell these five stocks, we would once
again incur the costs of the trades, which would be another 50USD or
150USD. To make the round trip (buying and selling) on these five
stocks it would cost us 100USD or 300USD, or 1\% or 3\% of our initial
deposit amount of 10,000USD. If our investments do not earn enough to
cover this, we have lost money by just entering and exiting positions
(the 10,000USD is far beyond what most small investors have to start
with!). In addition, hidden costs are the norm and most regulations
underlying investing should be explored further. 
\end{comment}


\begin{comment}
%\subsection{Keeping a balance between our ideas and worldwide distributed ideas}

  There are millions of ideas fluctuating out there. Do most of them
  go to extinction quickly? Diversifying portfolio in science and
  everyday life is a simple consequence of living in fluctuating (and
  unpredictable) environments. The closer we are at predicting
  fluctuations of several time series (including your ideas) at local
  and regional scales the better we know the ecosystem. Unfortunately,
  it is not easy to predict time series of a large number of
  interacting (ideally independent) species (or ideas or market
  products from stocks to bonds). Given we can not predict most of the
  ideas' trends, we should build a minimum understanding on how to
  investigate ideas and build a diversified portfolio with a balance
  between risk and reward. Basic questions will always remain when
  discussing about predicting the future and diversifying
  portfolios. For example, in a complex ecosystem, which is the best
  strategy under complete ignorance? And under complete information?
  Should we invest following a random walk \index{random walk}? Should
  we produce a portfolio with neutral risk \index{neutral risk}?
  \footnote{\url{https://en.wikipedia.org/wiki/A$_$Random$_$Walk$_$Down$_$Wall$_$Street}}

%\section{Current portfolio theory in economics and ecology}

Portfolio theory in economics has a long tradition
\citep{MarkowitzBook}. The theory is rooted in the concept of
efficient frontier\index{efficient frontier}. There are several
companies and packages offering codes in several languages to
calculate efficient
frontiers\footnote{\url{http://www.quantcode.com/}}$^{,}$\footnote{\url{https://github.com/JuliaQuant/PortfolioModels.jl}}$^{,}$\footnote{\url{https://www.wikinvest.com/account/portfolio/register}}$^{,}$\footnote{\url{https://d1so5k0levrfcn.cloudfront.net/SigFig\%20Investment\%20Methodology.pdf}}
. Most maths underlying portfoliio theory\index{portfolio theory} are
based in matrix correlation patterns\index{matrix correlation
  patterns}. In ecology, portfolio concept has also been used to
predict the number of coexisting species in landscapes with highly
fluctuating environments\footnote{Check references}.

%\subsection{Developing a highly diversified portfolio in time and space}

Given the basic maths underlying the portfolio theory, which are the
algorithms and models out there? Which one perform the best? Which is
the mixed of models to develop our investment strategy? Should we just
maximize reward and minimize risk (or volatility\index{volatility}) regardless
the frequency of trading? How to achieve it? How do frequency of
trading relate to maximum reward and minimum risk strategy? How to
connect a highly diversified portfolio with the knowledge of the
worldwide flow distribution of basic goods?\footnote{Check Bern
  principles}

%\newpage
%\section{Exploring de-centralized investments}

%\subsection{Centralized investments}
%The meaning of a centralized investment. Which are the pros and cons? Why centralized?

%\subsection{Decentralized investments and virtual currency}

The meaning of decentralized investment. The origin of bitcoin and
refs to invest in virtual currencies.

%\newpage
%\section{Open platform to investigate markets ($\mathcal{ROBHOOT}$)}

In the long term, $\mathcal{ROBHOOT}$ will be a semi-automated investment tool combining
access to high-resolution data for both centralized and decentralized
investments, and free-individual decisions in individual products. To achieve such a combination, we aim to
develop $\mathcal{ROBHOOT}$ in two stages. The first stage will be to
develop the free-access platform to have access to the high-resolution
data to investigate markets (Figures 1 and 2). The second stage will
be to develop the steps required to allow individuals to invest
(Figure 3). There are several bottlenecks to develop such
platform. Enumerating these bottlenecks to recognize our lack of
expertise is a key first step. At this stage, there are at least 8
bottlenecks.

%\subsection{Models}

%\begin{comment}
1) Él ha analisado series temporales de stocks del mercado brasileiro
con muestras a diferentes escalas temporales (semanales, diárias,
intraday de 10 minutos) 2) Él ha comparado modelos linales, no
lineales basados en maquinas de vectores de suporte, y modelos
híbridos.  3) Ha dicho que modelos híbridos generan mejores
predicciones de volatilidad que los demás modelos 4) No hay diferencia
significativa entre los modelos no lineales que él estudió y los
modelos lineales 5) La combinación de modelos que genera menos errores
es utilizar modelos *lineales* para predecir los valores esperados y
los modelos *hibridos* para predecir la volatilidad 6) Ha dicho que la
principal limitación de los modelos que ha estudiado es que los
errores (los componentes aleatorios) son muy altos cuando comparaods a
los componentes determinísticos.  Además de eso, la varianza de los
componentes aleatorios cambia a lo largo del tiempo, lo que hace mucho
dificil la predicción.  7) Ha dicho que los modelos de black-Litterman
utilizan la matriz de covarianza y posiblemente tendrán la misma
limitación en la predicción que los modelos que él ha estudiado en su
master (él no conocía el Black-Litterman, pero ha mirado el paper que
Carlos nos envió y me lo dijo).  8) El artículo que envié
anteriormente, que habla de la predicción de Bitcoin, también utiliza
la matriz de covarianza. Pero Gilmar cree que dicha predicción solo
fué posible porque ellos utilizan informaciones del "libro de
ofertas", que resulta en un valor entre -1 y 1. Si el valor es -1,
significa que todos en el mercado solo quieren vender.  Si el valor es
+1, siginifca que todos solo quieren comprar. Ha dicho que es tipo de
información es dificil para el mercado de stocks. Creo que allí
entramos nosotros! ;) 10) Por esa razón, él cree que es dificil hacer
predicciones fiables del comportamiento de stocks.  Por eso cree que
lo mejor es estudiar el riesgo de los stocks, en vez de la predicción.
11) Ha dicho que para su doctorado, él va a utilizar solamente modelos
lineales, porque ha dicho que lo que se gana con los modelos no
hibridos no le parece suficiente como para perder en la performance de
la simulación. Su objetivo es intentar mejorar la limitación que
mencioné anteriormente.
%\end{comment}

%\subsection{Pattern detection models}

%\subsubsection{Modern Portfolio Theory}

Explain briefly the main assumptions and the main albegra used. 
Mention the matlab and julia codes with access to quandl data.

%\subsubsection{Mixed Estimation Model: Black-Litterman model}

The Mixed Estimation Model was developed by Henri Theil in the early
1960's, but was applied to financial data by Fischer Black and Robert
Litterman in the early 1990's. The beauty of this model is that one
can blend a variety of views specified in different ways, absolute or
relative, with a given prior estimate to generate a new and updated
posterior estimate which includes all the views. The updated posterior
estimate should be centered more closely around the unknown mean, and
should also have a lower variance(higher precision) that either the
prior or conditional distribution. There are implementations in
matlab\footnote{\url{http://www.blacklitterman.org/intro.html}} and
new automated investment companies use the Black-Litterman model for
their automated
decisions\footnote{\url{https://www.betterment.com/resources/investment-strategy/why-you-should-invest-beyond-us-stocks/}}.
$\mathcal{ROBHOOT}$ uses the matlab version
from\footnote{\url{http://www.blacklitterman.org/intro.html}}
implemented in julia.


%\subsubsection{Neural network model}

%\subsubsection{Elliott waves model}


%\subsection{Dynamic or mechanistic models}

%\subsubsection{Brownian motion model (Random walk model)}



%\section{Bottlenecks}

%\subsubsection{Bottleneck 1: Data access algorithms}

Streaming data analysis will be the key when accessing the
data. Investment products datasets are too large to store in RAM and
are being generated in real time.


Bottleneck 1 would require to retrieve ID's from the Bloomberg open
symbology\footnote{Methods to convert each product between their
  ID's. For example, convert 9-digit CUSIP codes into ISIN codes
  \url{http://stackoverflow.com/questions/30545239/convert-9-digit-cusip-codes-into-isin-codes}}. This
is not just connecting algorithms to google or yahoo to have access to
interday data for a few markets and products (this is by far a trivial
problem). Data access has, at least, three steps: 1) ID database
access using open symbology; 2) link ID to the several stickers each
ISIN contains with a different sticker name in each different market
(ID or ISIN can be compared with the genome of each investment product
and the stickers would be the different varieties of a species in
space), and 3) connect the ID or ISIN to the time series at high
(intraday) or low (interday) resolution with price and volume
values. TODAY, THERE IS NO SERVER OFFERING THIS EASY ACCESS SERVICE
FOR FREE FOR ANY PRODUCT IN ANY MARKET.

%\subsubsection{Bottleneck 2: Pattern detection algorithms}

Bottleneck 2, pattern detection, includes several open problems but
they can mostly be reduced to one. This is to have a solid background
in stochastic portfolio theory \index{stochastic portfolio theory} to build a codebase \index{codebase}
from the most simple one investment product to several investments
products. The codebase should be ready to simulate the efficient
frontier for any given portfolio.

%\subsubsection{Bottleneck 3: Mechanistic model algorithms}

Bottlenecks in step 3 are those based in model development, selection
and inference. Complex models will be difficult to analyze using
Approximate Bayesian computation methods to develop diversified
investment strategies. Thus, combining our own (complex) decisions
with the available algorithm and methods should allow to simplify the
models and run ABC to compare them before developing the strategies to
invest.

%\subsubsection{Bottleneck 4: Validation and ABC algorithms}

%\subsubsection{Bottleneck 5: Automated decision algorithms}

%\subsubsection{Bottleneck 6: API development, cloud access and counting metrics}

%\subsubsection{Bottleneck 7: Open platform in a highly regulatory market}

%\subsubsection{Bottleneck 8: Security and encryption methods}
\end{comment}


\newpage
\section{Acknowledgments}


\newpage
\bibliographystyle{evolution}
\bibliography{ref}

\newpage

\section{Tables}


\newpage

\section{Figures}

%Figure 1: TRADE-OFFS: Individual decisions at high cost and automated decision at low cost (Why not both?)


%Figure 2: ROBHOT: Steps to develop ROBHOT
\begin{figure}
\vspace{-1 in}
\begin{center}
\includegraphics[scale=0.8]{EasyFlowChart.pdf}
\end{center}
\caption{Flow chart summarizing the steps to develop $\mathcal{ROBHOOT}$}
%\label{}
\end{figure}
\newpage

%Figure 3: ROBHOT: Steps to develop ROBHOT
\begin{figure}
\vspace{-1 in}
\begin{center}
\includegraphics[scale=1.25]{EasyFlowChartBottlenecks.pdf}
\end{center}
\caption{Bottlenecks to overcome to develop $\mathcal{ROBHOOT}$}
\end{figure}
\newpage

%Figure 3: ROBHOT: Steps to develop ROBHOT
%\begin{figure}
%\vspace{-1 in}
%\begin{center}
%\includegraphics[scale=0.8]{EasyFlowChartOPTIIMA.pdf}
%\end{center}
%\caption{Flow chart summarizing the steps to develop ROBHOT (phase II)}
%\label{}
%\end{figure}

\printindex
\end{document}
