\documentclass[dvipsnames]{article}
\thispagestyle{empty}
\usepackage{xcolor}
\usepackage{tikz}

\usetikzlibrary{arrows,
                positioning,
                shapes.geometric}
              \colorlet{ColorPink}{red!30}
              \colorlet{ColorPink2}{red!50}
              \colorlet{ColorGreen}{OliveGreen}
         
\begin{document}
   \begin{figure}[h!]
     % \centering
     \vspace{-1 in}
     
    \hspace{-1.75 in}\begin{tikzpicture}[
    node distance=10mm and 10mm,
% Define block styles
     base/.style = {text width=28mm, minimum height=20mm, align=center},
startstop/.style = {base, rounded corners, fill=yellow!90,minimum height=20mm,text width=65mm},
  io/.style args = {#1/#2}{rectangle, rectangle stretches body,
                    draw, fill=blue!20,
                    minimum width=#1, minimum height=#2,
                    text width =\pgfkeysvalueof{/pgf/minimum width}+11*\pgfkeysvalueof{/pgf/inner xsep},
                    align=center},
                  processK/.style = {base,text width=95mm, minimum height=130mm,fill=white!80},            
                  processL/.style = {base,text width=85mm, fill=pink!50},
                  processX/.style = {base,text width=75mm, minimum height=40mm,fill=ColorPink2},
                  processE1/.style = {base,text width=5mm, minimum height=5mm,fill=Black},
                  processE2/.style = {base,text width=5mm, minimum height=5mm,fill=Black},
 processS/.style = {base,  text width=105mm, minimum height=175mm, align=center, fill=ColorPink},
 decision/.style = {diamond, aspect=1.5,
                    base, fill=green!30},
                  arrow/.style = {thick,-stealth}            
    ]

    
    \node (LABX)    [processX, center=of LAB1]  {};
    \node (LABE1)    [processE1, above left=of start]  {};
    \node (LABE2)    [processE2, above right=of start]  {};
                        \node (start)  [rectangle,startstop]                 {\Large {{\bf OPEN RESEARCH NETWORK}}};
                        \node (LAB1)    [processL, below=of start]  {{\Large {\bf WHY?}}};
                        
%-----------------------------------------------------------%
\begin{scope}[io/.default = 10em/13mm]
\node (in1) [io, below left=of LAB1]    {{\bf GLOBAL\\ RESEARCH\\ KNOWLEDGE\\ GAP:\vspace{0.25 in} \\CENTRALIZATION\\ BIAS\\ ERROR-PRONE\\ NONREPRODUCIBILITY\\ AND\\ LACK OF INCENTIVES}};
%\node (in2) [io, below =of in1]        {{\bf CENTRALIZATION, BIAS, ERROR, NONREPRODUCIBILITY AND LACK INCENTIVES}};
\node (in3) [io, below right=of LAB1]   {{\bf REPRODUCIBILITY AND OPEN GLOBAL\\ SYSTEM SCIENCE\\ REPORTS \vspace{1.32 in}}};
    \end{scope}
%-----------------------------------------------------------%

    \node (LAB2)    [processS, below=of LAB1,label={[shift={(0,-17)}] {\Large {{\bf FULLY AUTOMATED TECHNOLOGY}}}}];
    
    %\node (LAB5)    [processS, below=of LAB1,label={[shift={(0,-15)}] {{\bf KNOWLEDGE GRAPHS}}}];
    %\node (LAB6)    [processS, below=of LAB1,label={[shift={(0,-15)}] {{\bf DEEP LEARNING NETWORKS}}}];

    %\node (LAB4)    [processS, below=of LAB2];
    \node (LAB3)    [processL, below=of LAB1]       {{\Large {\bf HOW?}}};
     \node (LAB2)    [processK, below=of LAB3,label={[shift={(0,-12.5)}] {{\bf DECENTRALIZED KNOWLEDGE GRAPHS}}}];
\node (testdrive)   [processL, below=of LAB3]   {{\Large {\bf UNIVERSAL ETLs}}};
\node (evaluation)  [processL, below=of testdrive]  {{\Large {\bf BAYESIAN SPACE MODELS}}};
\node (comparison)  [processL, below=of evaluation] {{\Large {\bf ANIMATION SPACE}}};
\node (stops)   [startstop, below=of comparison]    {\Large {{\bf REPORTING}}};
\node ()    [below=of startstop,label={[shift={(0,-20)}] {{\bf IN DEEP LEARNING NETWORKS}}}];
%-----------------------------------------------------------%
%\draw [arrow,line width=1mm]   (start)   edge (LAB1)
                %(LAB1)    edge (in2) 
                %(in2)     edge (LAB2)           
%                (LAB3)    edge (testdrive)
%                (testdrive)   edge (evaluation)
%                (evaluation)  edge (comparison)
%                (comparison)  to (stops);
%-----------------------------------------------------------%
%                \draw [arrow,red,line width=1.5mm] (LAB1) -| (in1) node[midway,above]{{\bf REDUCE}};              % around trapecium
%                \draw [arrow,OliveGreen,line width=1.5mm] (LAB1) -| (in3) node[midway,above]{{\bf INCREASE}};
                %\path[->,draw=red,line width=1mm,fill=red]  (LAB1) -- (in1);
                %\draw [<->,red] (in1) -- (in2); 
%\draw [->,green] (in1) -- (LAB4);
%\draw [arrow] (in2) |- (LAB3);
%\draw [<-->,arrow,line width=1mm] (in1) -- (LAB2);
%\draw [<-->,arrow,line width=1mm] (in3) -- (LAB2);
\end{tikzpicture}
    %\caption{Flow-chart}
    %\label{figure3111111}
\end{figure}

\end{document}
