\documentclass[twocolumn]{article}

\usepackage{hyperref} % For hyperlinks in the PDF
\usepackage{authblk}  % To add author affiliations
\usepackage{graphicx}
\usepackage{cleveref}
\usepackage{tcolorbox}
\newtcolorbox{mybox}{colback=gray!5!white,colframe=red!75!black}
\usepackage{setspace}

% Personal command
\graphicspath{ {../} }
\newcommand{\beginsupplement}{%
\setcounter{table}{0}
\renewcommand{\thetable}{S\arabic{table}}%
\setcounter{figure}{0}
\renewcommand{\thefigure}{S\arabic{figure}}%
}

%----------------------------------------------------
%	TITLE SECTION
%----------------------------------------------------

\title{{Title: Deep process-based learning networks in Biodiversity research}}
\author[]{}
% \author[*,1]{Ali R. Vahdati}
% \affil[1]{Department of Anthropology, University of Zurich, Zurich, Switzerland}
% \affil[*]{Corresponding author. Email: ali.rezaee@aim.uzh.ch, zolli@aim.uzh.ch}

% \setcounter{Maxaffil}{0}
% \renewcommand\Affilfont{\itshape\small}

\date{} % Leave empty to omit a date. Add \today for current date.

%----------------------------------------------------

\begin{document}

\maketitle


\section*{Summary}
We are in a enthralling scientific era. We have the computer power,
the open-source tools, the know-how in many highly specialized fields
and the team capabilities to integrate Earth science and Biodiversity
research in open and decentralized automated research platforms. We
are in a period where novel analytical methods and data are being
fussioned at an incredible speed to decipher the complexity and
feedbacks between the Earth system and the diversity of life. Yet, we
are in a massive human-driven biodiversity extinction with large
uncertain consequences for Earth climate, life conditions and the
stability of Earth (Figure 1). This combination of an enthralling
scientific era and rapid global change put us in an edge to team up to
go beyond our disciplinary boundaries to contrast scenarios accounting
for feedbacks between the Earth system and Biodiversity (Figure
2). For this to happen we need to connect fundamental and applied
science (Figure 3b) and one way to do it is throughout distributed
open research platforms to provide informaton for management forums in
applied conservation and sustainability centers. This proposal aims to
develop a distributed open-source automated research platform to
integrate multiple databases into Biodiversity dynamics and function
scenarios taking into account the interdependencies among biological
levels and scales in ecological and evolutionary networks (Box 1 and
Figures 3 and 4).

\newpage
\setcounter{page}{0}
\clearpage
\pagebreak
\section*{Deep process-based learning networks in Biodiversity research}

\noindent Biodiversity is declining globally at unprecedented rates (\Cref*{fig:1}).
Despite the importance of biodiversity for persistence of all species, including
humans (REF), we do not yet understand the interactions between biodiversity and 
Earth system - physical, biological, and chemical properties of the Earth as an
integrated system. Biological systems are composed of many layers, and they can
contain interdependent hierarchies and feedbacks with interacting
learning entities within and between the layers (Figure 2). Understanding interactions between biodiversity and Earth system requires analysis of life at different levels,
from genes to individuals to populations and to communities (\Cref*{fig:2}). There is an immense
 detailed knowledge at each of the levels and scales studied in biodiversity
research. Yet, such data and knowledge is not sufficiently integrated to provide
 a holistic understanding of feedbacks between biodiversity and Earth system.
We aim to creating an automated research platform that integrates different pieces
of data, information, and knowledge of biodiversity research from different levels
into a single framework. % TODO more detail of the objective and its impact

\begin{figure}[htp]
  \centering
  \includegraphics[height=0.010\maxdimen, width=\linewidth]{{Figure1}.png}
	\caption{
		\textbf{Decline of biodiversity across the globe}.
       Map showing the remaining populations of native
       species across many taxa as a percentage of their original
       populations. Blue areas are within proposed safe limits, and
       red areas are beyond these limits. For further information
       please check the original work at
       \url{http://www.nhm.ac.uk/discover/news/2016/july/biodiversity-breaching-safe-limits-worldwide.html}.
		}
	\label{fig:1}
\end{figure}

% TODO what is an automated research platform and why is it important?
Despite the rapid development of automated research platforms integrating different aspects of scientific cycle \cite{NakamotoT,BigQuery,AutomatedStatistician,Modulus,GAI,Iris,easeml}, distributed open-source automated research platforms in biodiversity research are still at an incipient stage (REF).

One reason of still being at an incipient stage is that
most methods in biodiversity research have been
considered classically as distinct fields. However, the current
scientific ecosystem is at a stage where merging methods from
distinct fields is radically transforming the discipline boundaries,
the reproducibility of science, and our prediction/understanding
power \cite{Reichsteietal2019}.
% TODO: explain why deep learning.
Many of the recent approaches applying
deep learning methods in ecology and evolution have mostly focused at
one level of biological organization \cite{Sheehan2016}. While this might produce
additional gain in detailed knowledge at each level, it remains
unknown how many layers are needed for predicting the
consequences of feedbacks between Earth system and biodiversity.

To gain predictive and understanding power in biodiversity
research we are going to need to merge distinct databases into hybrid
deep process-based learning methods accounting for many layers and the
topology of the interactions within and between the
layers \cite{Melian2018}. Many methods
from data science and biological systems share fundamental properties
(i.e., network-like patterns, multiple layers, etc). Yet, the full
potential of these shared properties have not been sufficiently
explored. We will
integrate different biological layers into a platform to explore
contrasting scenarios of Biodiversity dynamics accounting for
interdependencies and feedbacks within and between layers (Box 1 and
\Cref*{fig:3,fig:4}).

\begin{figure}[htp]
  \centering
  \includegraphics[height=0.018\maxdimen, width=\linewidth]{{Figure2-2}.pdf}
	\caption{
  \small{\textbf{Biodiversity is hierarchically
         structured} yet inferring interdependencies among the levels
       developing hybrid deep-process based learning approaches to
       predict the consequences of biodiversity decline remains poorly
       studied. A) Biodiversity has been studied mostly considering
       independent levels, from genes, traits and populations to
       communities and ecological networks. B) Biodiversity
       represented as interdependent levels accounting for feedbacks
       from genes and traits, and from traits and populations to
       communities. It remains unknown which of these two scenarios
       best predict current trends in Biodiversity decline and its
       consequences for Earth climate, life conditions and the
       stability of Earth.}
		}
	\label{fig:2}
\end{figure}

\begin{mybox}\begin{singlespace}
{\bf{Box 1. Deep process-based learning networks in Biodiversity research}}\\
\begin{small}
  We will implement a multilayer approach to generate process-based
  species distribution maps accounting for interdependent biological
  networks. Each layer will be parametrized taking advantage from the
  integration of biodiversity datasets. Most data in biodiversity are
  collections of small data. In areas such as species ranges and
  species interactions, there is a large amount of data, but only a
  relatively small amount of data for each gene, phenotype, individual
  or trophic interaction. To customize predictions accounting for
  interdependent biological levels we will use a formalism considering
  the heterogeneity at individual level, with its inherent
  uncertainties, and to couple the individual level together in a
  hierarchy scaling from genes to phenotypes, populations, communities
  and species ranges, so that information can be borrowed from other
  similar levels across the landscape in the absence of empirical
  estimations. We will implement a multilayer approach using
  hierarchical Bayesian neural networks\cite{Ghahramani:2015}. The outputs of the
  multilayer approach will generate a biodiversity distribution map
  for many interacting species that can be evaluated against the
  empirical patterns.

  We will contrast two scenarios to explore the best one fitting the
  empirical patterns. The first scenario will simulate independent
  levels considering modularity within- and between-layers (i.e., a
  highly modular pleiotropy matrix determining the genotype-phenotype
  map and a highly modular within- and between-species interactions
  with most interactions weak or zero across the landscape.) Such
  scenario will produce a non- or weakly-interactive species
  biodiversity map. The second scenario will account for feedbacks
  among layers. We will explore a range of topologies from
  bidirectional recurrent neural networks (BRNN) to feedforward neural
  networks (FNN) and reinforcement learning in unknown and fluctuating
  environments (RL) \cite{Schmidhuber:2015}. Such scenario will produce an (strongly)-interactive
  species biodiversity map. We will disturb both scenarios following
  random and non-random disturbance regimes (i.e., removing specific
  interactions, abundances and habitats) and will quantify responses
  to disturbances using a variety of metrics, from biodiversity to
  functional metrics \cite{Melian2018}.
\end{small}
\end{singlespace}
\end{mybox}


\begin{figure}[htp]
  \centering
  \includegraphics[height=0.018\maxdimen, width=\linewidth]{{Figure3-2}.pdf}
  \caption{
    \textbf{Prediction and understanding power
         map}. This figures shows a cartoon of a prediction power map
       (top), an understanding power map (middle), and a
       predicting-understanding power map (bottom). x- and y-axis
       represent data-based inference (i.e., gradient of AI methods
       from low (left) to high (right) predictive power) and
       process-based inference (i.e., gradient of process-based
       methods from low (bottom left) to high (top left) understanding
       power). The gradient of predicting power map (top) shows a hot
       spot red area in the bottom right highlighting the region where
       AI methods best predict the empirical data. The gradient of
       understanding power map (middle) shows a hot spot red area in
       the top left highlighting the region where the best mechanistic
       understanding occur. The predicting-understanding power map
       (bottom) shows the sum of the two previous maps highlighting a
       red hot spot where the best synthesis research joining
       predicting and understanding power of the empirical data might
       occur.
    }
  \label{fig:3}
\end{figure}

\begin{figure}[htp]
  \centering
  \includegraphics[]{{Figure4}.pdf}
	\caption{
		\textbf{Figure 4: Prototyping a distributed and open automated research
    platform:} {\bf a)} Our initial prototype will contain five
  layers (this is not an exhaustive number. Some might be merged
  and others, like reporting generation, can be introduced): Data
  Integration, Complexity reduction, Pattern-process inference,
  Validation, and Visualization. Nodes and links represent
  algorithms and interactions between two algorithms,
  respectively. The inter-layer interactions will be implemented
  using the Renku-SDSC
  platform \cite{Renku}. The
  intra-layer interactions will be developed initially in julia
  language (other languages will come into play during the
  development of each layer). {\bf b)} A julia-computing-language
  prototype of an automated research platform. Nodes and links in
  each layer represent julia packages and interactions between
  two packages, respectively. The figure shows the julia packages
  to be used for the Data integration layer containing the
  packages "Retriever.jl" ({\bf Re}), "Query.jl" ({\bf Qu}),
  "MySQL.jl" ({\bf My}), "SQlite.jl" ({\bf lite}), and
  "DataFrames.jl" ({\bf df}). This cartoon representing many
  intra- and inter-layer connections might be helpful to show the
  vision of the platform. For example, the path taken to solve a
  specific intra- or inter-domain (fundamental or applied)
  question can be quantified by many metrics each producing a
  distribution of automated solutions across many nodes in a
  distributed and open network, the Robhoot Open Network
  (RON). This distribution can be analyzed to quantify properties
  as robustness, reproducibility and bias of a fundamental or
  applied solution.
		}
	\label{fig:4}
\end{figure}

\clearpage
\pagebreak
\bibliographystyle{abbrv}
\bibliography{ref}



\end{document}