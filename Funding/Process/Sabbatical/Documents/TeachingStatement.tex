\documentclass[a4paper,12pt]{article}
\usepackage{fancyhdr}
\pagestyle{fancy}
% with this we ensure that the chapter and section
% headings are in lowercase.
\renewcommand{\chaptermark}[1]{\markboth{Researh Statement}{}}
\renewcommand{\sectionmark}[1]{\markright{\thesection\ Research Statement}}
\fancyhf{} % delete current header and footer
\fancyhead[LE,RO]{\bfseries\thepage}
%\fancyhead[LO]{\bfseries\rightmark}
\fancyhead[RE]{\bfseries\leftmark}
\renewcommand{\headrulewidth}{0.5pt}
%\renewcommand{\footrulewidth}{0pt}
%\addtolength{\headheight}{0.5pt} % space for the rule
\fancypagestyle{plain}{%
   \fancyhead{} % get rid of headers on plain pages
   \renewcommand{\headrulewidth}{0pt} % and the line
}

\usepackage{latexsym}
\usepackage{caption2}
\usepackage{graphicx}
\usepackage{natbib}
\usepackage[latin1]{inputenc}
\usepackage{color}
\usepackage{array}
\usepackage{longtable}
\usepackage{calc}
\usepackage{multirow}
\usepackage{hhline}
\usepackage{ifthen}
\usepackage{lscape}

\oddsidemargin 0.5cm

\textwidth 16cm

\textheight 20cm

%\pagestyle{empty}

\vspace{0.1 in}

\hspace{-0.3 in}
\markboth{Teaching statement -- CJ Meli\'an}{Teaching statement -- CJ Meli\'an} 

 \centerline{Carlos J. Meli\'an}\\
\hspace{-0.4 in} \centerline{Center for Ecology, Evolution and Biogeochemistry}\\
\centerline{EAWAG, Seestrasse 79, Kastanienbaum CH-6047, Switzerland.}
\vspace{0.5 in}


\begin{document}

{\Large Teaching Statement} 

\vspace{0.25 in} During my scientific career I had the occasion of
teaching in different countries and at different education levels. In
Spain I gave lectures of ecological modeling in undergraduates courses
during the beginning of my PhD. in the University of Alcala,
Madrid. During my PhD. I gave several lectures to PhD students on
graph theory and its applications to biological networks in the
Spanish Research Council (CSIC).

During my postdoctoral research at the National Center for Ecological
Analysis and Synthesis (NCEAS) I have been involved in the program
$Kids$ $Do$ $Ecology$ during the last three years teaching ecology to
kids with different
ages\footnote{http://www.nceas.ucsb.edu/nceas-web/kids/kdesb/2008franklin.html}. I
have been actively participating in different teaching programs that
attempt to develop strategies for education in ecology, diversity and
sustainability. I participated in the ``SEEDS'' Leadership Meeting
(ESA), in Albuquerque, New Mexico, February 2009
\footnote{http://www.esa.org/seeds/} to discuss about bringing big to
light ideas in ecology and evolution. I also organized a course at
NCEAS in 2009 about ``Theoretical ecology and data-driven synthesis.''

During my tenure-track at Eawag, Switzerland, and now as a lecturer at
the University of Bern I have been co-teaching in different lectures
and courses with Prof. Ole Seehausen and Prof. Barbara Taborsky. I now
teach in one semester to $3^{rd}$ year biology students introducing
them to scientific analysis, the cycle of the scientific process and
the many skills required to become a scientist.

Thanks to those experiences in different countries I have developed
some teaching skills. I am confident that students appreciated my
teaching style. Because most of the topics I have teached-discussed
require effort from the students, I think that it is essential to
clearly state why these techniques are useful for the real world. I
provide examples and case studies together with bringing researchers
to present more in depth topics that motivate students to discuss and
allow them to apply what they have learned in their careers.
\newpage

From my previous experience an important point in the teaching process
is that students should be exposed to the diversity of methods and
solutions and the pros/cons of each one for any given problem. I also
complement the lectures with practicals communicating the broad set of
online open-source tools available for scientific computing in general
and for biodiversity research in ecology and evolution in
particular. Methods in scientific computing applied to biodiversity
are rich and diverse and they are expanding and merging very fast with
novel methods coming from many different disciplines at a fast
pace. This evolution is essentially driven by the accumulation of
larger and more accurate and complex data sets (i.e., NGS,
paleontological, historical, environmental, and contemporary in the
context of ecology and evolution). This implies that techniques with
different assumptions, reliability and power should be clearly
understood to be tested with a large amount of heterogeneous data.

I can teach fundamentals of graph theory, multilayer networks, Neutral
Bayesian networks, Bayesian statistics with attention to inference,
ABC, Bayes factors, optimization algorithms, and stochastic processes
in networks. On the more applied and specialized side, I can focus on
networks in ecology and evolution (i.e., eco-evolutionary networks)
and spatiotemporal dynamics of biodiversity, stochastic dynamics in
networks with attention to genetic and ecological drift, and
fluctuating selection in networks. I have been using always
open-source tools like julia, sage and octave for the practicals with
special attention in notebooks like Jupyter to facilitate
reproducibility in the classroom.

  \vspace{0.2 in}

Yours Sincerely,
\\
\\
Carlos J. Meli\'an,  \hspace{0.2 in} Kastanienbaum, 1 March 2019\\
\end{document}



