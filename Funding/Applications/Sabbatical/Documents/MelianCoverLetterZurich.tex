\documentclass[a4paper,12pt]{letter}

%\usepackage{latexsym}
\usepackage{caption2}
%\usepackage{graphicx}
%\usepackage{natbib}
\usepackage[latin1]{inputenc}
\usepackage{color}
\usepackage{array}
%\usepackage{longtable}
%\usepackage{calc}
%\usepackage{multirow}
%\usepackage{hhline}
%\usepackage{ifthen}
%\usepackage{lscape}

\oddsidemargin 1cm

\topmargin 0.75cm 
\textwidth 15cm

\textheight 21cm

\pagestyle{empty}

\begin{document}
\vspace{-4 in}

To Prof. Dr. Roland Sigel,\\
Dean of the Faculty of Science, University of Z\"urich,\\
Winterthurerstrasse 190, CH-8057 Z\"urich, Switzerland\\
\\
\\
Dear Prof. Roland Sigel,\\
\vspace{0.05 in}

Please consider my application for the associate or full professor
position in ``Data Science for Sciences'' at the University of Zurich.

My interest in connecting data science to process-based theory to
decipher the origin and coexistence of biodiversity began while I
studied Biology and Environmental Sciences from 1996-1999 in Madrid,
Spain. Through several undergraduate research projects, I became
interested in fussioning environmental sciences with the ecology and
evolution of ecosystems and began working on my PhD. with Dr. Jordi
Bascompte at the Do\~nana Biological Station, Seville, Spain. I
completed my PhD. in 2005 on ``The Structure and Dynamics of
Ecological Networks'' focused on fussioning modern data-driven methods
with theory in networks.

I was a postdoctoral fellow at the National Center for Ecological
Analysis and Synthesis (NCEAS) at University of California in Santa
Barbara from late 2005 to 2010. NCEAS strongly influenced me to learn
from data-scientists to fussion small, medium and large environmental
and ecological data with broad and general modeling frameworks.

I got a second research fellowship (unrestricted gift) from Microsoft
research to continue developing novel methods at NCEAS to bridge
heterogeneous ecological data with process-based frameworks scaling
from individuals to ecosystem and biodiversity patterns. During this
period I introduced my science in the open-source scientific software
arena with special attention to reproducible research, data
digitalization and new interdisciplinar collaborations. I obtained a
tenure-track position at Eawag-ETH-Domain, Switzerland in 2010 and the
tenured in 2015. I am currently a lecturer at University of Bern where
I teach since late 2013.

I am particularly interested in a faculty position in a institute with
researchers who use/develop data-driven science bridging disciplines
and span across multiple spatiotemporal scales. My science has been
evolving to fully integrate data, novel statistical methods from
computer science, optimization, Bayesian statistics, Bayesian neural
networks and stochastic processes to understand the biodiversity
dynamics and their response to multiple human and non-human
disturbances integrating different biological levels and spatial
scales.

I have been actively leading international workshops in biodiversity
science and reproducible research ({\em CV.Melian.pdf}), teaching
undergraduates and PhD courses in three countries ({\em
  CV.Melian.pdf}), published 30 articles in domain and
interdisciplinar journals, and obtained funding as a PI from five
countries ({\em Publications.Funding.Melian.pdf}.)

The following are the three potential referees, along with their
complete institution address, email, and phone numbers:
\\
\item {\bf Prof. Jordi Bascompte}, Department of Evolutionary Biology and Environmental Studies\\
University of Zurich, Switzerland\\
e-mail: jordi.bascompte@uzh.ch\\
phone: +41 44 635 6126\\
\item {\bf Prof. Stefano Allesina},\\ 
Computation Institute, Department of Ecology \& Evolution,\\
University of Chicago, USA.\\
e-mail: sallesina@uchicago.edu\\
phone: +1 773 702 7825 \\
\item {\bf Prof. Miguel B. Araujo} (email:),\\
  Natural History Musseum, Spanish Research Council, Madrid, Spain.\\
  e-mail: maraujo@mncn.csic.es\\
  phone: +34 91 411 1328\\
  \\
  \\
  Please find enclosed my curriculum vitae, my research and vision
  statement outlining major unsolved problems in Biodiversity
  research, my teaching statement, and my publications and funding
  summary with reference to my three top papers making a short
  statement explaining the choice.
  \\
  \\
  Thank you for your consideration.  \vspace{0.2 in}
  \\
  Sincerely Yours,
  \\
  \\
  Carlos J. Meli\'an,\hspace{1.75 in}  Kastanienbaum, 1 March, 2019\\
  Phone: +41 58 765 2208\\
  e-mail: carlos.melian@eawag.ch\\
  https://www.eawag.ch/en/aboutus/portrait/organisation/staff/profile/carlos-melian/show/
\end{document}
