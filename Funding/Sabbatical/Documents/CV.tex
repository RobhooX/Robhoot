\documentclass[12pt]{article}
\usepackage{fancyhdr}
\pagestyle{fancy}
% with this we ensure that the chapter and section
% headings are in lowercase.
\renewcommand{\chaptermark}[1]{\markboth{Researh Statement}{}}
\renewcommand{\sectionmark}[1]{\markright{\thesection\ Research Statement}}
\fancyhf{} % delete current header and footer
\fancyhead[LE,RO]{\bfseries\thepage}
%\fancyhead[LO]{\bfseries\rightmark}
\fancyhead[RE]{\bfseries\leftmark}
\renewcommand{\headrulewidth}{0.5pt}
%\renewcommand{\footrulewidth}{0pt}
%\addtolength{\headheight}{0.5pt} % space for the rule
\fancypagestyle{plain}{%
   \fancyhead{} % get rid of headers on plain pages
   \renewcommand{\headrulewidth}{0pt} % and the line
}
\usepackage{graphics}
\usepackage{latexsym}
\usepackage{caption}
\usepackage{flafter}
\usepackage{amssymb,amsmath}
\usepackage{eurofont}
\oddsidemargin 0.8cm

\textwidth 16cm

\textheight 20cm

\begin{document}

\date{}
\maketitle
\baselineskip=6.5 mm
\vspace{-1.5 in}
\centerline{\footnote{Last update April 1, 2010}}
\fbox{\parbox{0.95\linewidth}{\centerline{\Large \em Curriculum vitae}\\
\vspace{0.2 in}
\author{\centerline{{\Large Carlos J. Meli\'an, Ph.D.}}\\
\centerline{National Center for Ecological Analysis and Synthesis}\\
\centerline{University of California-Santa Barbara, Santa Barbara, CA 93101}\\
\centerline{Webpage: http://www.nceas.ucsb.edu/$\sim$ melian/}\\
\centerline{E-mail: melian@nceas.ucsb.edu, Phone: +1-805-892-2527, Fax: +1-805-892-2510}}}}
\vspace{0.2 in}

\begin{flushleft}
\section{ Education}
{\bf 2005} Ph.D. in Biology and Environmental Sciences. University of Alcal\'a,\\ Madrid and Biological Do\~nana Station,
Seville, Spain.\\ {\em Dissertation title: On the Structure and Dynamics of Ecological Networks.}\\
Dr. Jordi Bascompte, advisor.\\
{\bf 1998} B.S. in Biology and Environmental Sciences. University of Alcal\'a, Madrid, Spain.\\
{\bf 1994} B.S. in Economics. University of Las Palmas, Canary Islands, Spain.
\section{Awards and Fellowships}
\\
{\bf 2008} Postdoctoral fellowship granted by Microsoft Research Ltd. (Unrestricted Gift).\\
 {\em National Center of Ecological Analysis and Synthesis}, University of California, USA.\\
{\em Title: Unifying Theories of Molecular, Community and Network Evolution.}\\
{\em Period: 2 Years, 200,000\$}.\\
{\bf 2005} Postdoctoral fellowship granted by the {\em National Center of Ecological Analysis and Synthesis}, University of California, USA.\\
{\em Title: The Evolution of Behavior and the Structure of Complex Ecological Networks.}\\
{\em Period: 3 Years, 120,000\$}.\\

%\newpage
{\bf 2001-2004} PhD. fellowship granted by the Spanish Ministry of Science and Technology.\\
{\em Title: On the Structure and Dynamics of Ecological Networks.}\\
{\em Period: 4 Years, 42,000\euro}.\\
{\bf 2003} Environmental sciences young researcher award. Spanish Environmental Sciences Association.\\
{\bf 1999} Student research fellowship. Ministry of Foreign Affairs. Spanish Agency for International Cooperation. University of Granma, Bayamo, Cuba.\\
{\em Title: Linking Agroecology, Conservation and Theoretical Ecology.}\\
{\em Period: 4 Months, 3000\euro}.\\
{\bf 1997} Student research fellowship. Department of Ecology, Alcala University, Madrid.\\
{\em Title: Complex Systems: Theory and its Application to Ecology.}\\
{\em Period: 1 Year, 2500\euro}.\\
%\newpage
\section{Research}

\fbox{\parbox{0.95\linewidth}{\centerline{ISI Web of Knowledge (Melian CJ*)}\\
\centerline{Sum of times cited: 426}\\
\centerline{Average citations per item: 38.7}\\
\centerline{h-index: 8}}}

\subsection{Research papers (International Journals)}

\begin{enumerate}
\item Davis, J., Borda de \'Agua, L., and Allen, D. P., {\bf Meli\'an,
    C. J.} (2010). Neutral Biodiversity Theory Can Explain the
  Imbalance of Phylogenetic Trees, Evolution, {\em In Press}.
\item {\bf Meli\'an, C. J.}, K\v{r}ivan, V., and Star\'y,
  P. (2010). Dynamics of a Individual-Based Metaweb at Local and Macroecological Scales,
  {\em In Review}.
\item {\bf Meli\'an, C. J.}, Alonso, D., Allesina, S., Condit, R. S.,
  and Etienne, R. S. (2010). A Biodiversity Theory with Genetic
  Speciation, The American Naturalist {\em In Review}.
\item {\bf Meli\'an, C. J.}, Vilas, C., Bald\'o, F., Enrique
  Gonz\'alez-Orteg\'on, E., Drake, P., and Williams, R. J. (2010). A
  Neutral Evolution-Speciation Test for Individual-Based Food Webs,
  {\em In Review}.
\item {\bf Meli\'an, C. J.}, Alonso, D., V\'azquez, D. P., Regetz, J.,
  and Allesina, S. (2010). Frequency-dependent Selection Predicts
  Patterns of Radiations and Biodiversity, PLoS
    Comp. Biol., {\em In Press}.
\item Carnicer, J., Jordano, P., and {\bf Meli\'an, C. J.} (2009). The Temporal Dynamics of Resource Use by Frugivorous Birds: A Network Approach. {\em Ecology}, 90:1958-1970.\\
\item {\bf Meli\'an, C. J.}, Bascompte, J., Jordano, P., and K\v{r}ivan, V. (2009). Diversity in a Complex Ecological Network with Two Interaction Types. {\em Oikos}, 118:122-130.\\
\item Fortuna, M. A., and {\bf Meli\'an, C. J.} (2007). Do Scale-free Regulatory Networks Allow more Expression than Random Ones?. {\em Journal of Theoretical Biology}, 247:331-336.\\
\item V\'azquez, D. P., {\bf Meli\'an, C. J.}, Williams, N., Bl\"uthgen, N., Krasnov, B. R., and Poulin, R. (2007). Species Abundance and Asymmetric Strength in Ecological Networks. {\em Oikos}, 116:1120-1127.\\
\item Penteriani, V., Fortuna, M. A., {\bf Meli\'an, C. J.}, Otalora, F., and Ferrer, M. (2006). Can Prey Behavior Induce Spatially Synchronic Aggregation of Solitary Predators? {\em Oikos}, 113:497-505.\\
\item Bascompte, J. and {\bf Meli\'an, C. J.} (2005). Simple Trophic Modules for Complex Food Webs. {\em Ecology}, 86:2868-2873.\\
\item Bascompte, J., {\bf Meli\'an, C. J.}, and Sala, E. (2005). Interaction Strength Motifs and the Overfishing of Marine Food Webs. {\em Proceedings of the National Academy of Sciences of the USA}, 102:5443-5447.\\
\item {\bf Meli\'an, C. J.} and Bascompte, J. (2004). Food Web Cohesion. {\em Ecology}, 85:352-358.\\
\item Bascompte, J., Jordano, P., {\bf Meli\'an, C. J.}, and Olesen, J. M. (2003). The Nested Assembly of Plant- Animal Mutualistic Networks.\\ {\em Proceedings of the National Academy of Sciences of the USA}, 100:9383-9387.\\
\item {\bf Meli\'an, C. J.} and Bascompte, J. (2002). Complex Networks: Two Ways to Be Robust? {\em Ecology Letters}, 5:705-708.\\
\item {\bf Meli\'an, C. J.} and Bascompte, J. (2002). Food Web Structure and Habitat Loss. {\em Ecology Letters}, 5:37-46.\\

\subsection{Book chapters and invited book reviews}
\item {\bf Meli\'an, C. J.} (2006). Plant-Pollinator Interactions: What Are They Telling Us About Community Assembly, Conservation, and Ecosystem Management?. {\em Ecology}, 87:2683-2685.\\
\item Sabo, J. L., Beisner, B. E., Berlow, E. L., Cuddington, K., Hastings, A., Koen-Alonso, M., McCann, K., {\bf Meli\'an, C. J.}, and Moore, J. (2006). Predicting measurable food web properties with minimal detail and resolution. In Dynamic Food Webs: Multispecies Assemblages, Ecosystem Development and Environmental Change. de Ruiter, P., Wolters, V., and Moore, eds. Academic Press.\\
\item {\bf Meli\'an, C. J.}, Bascompte, J., and Jordano, P. (2005). Spatial Structure and Dynamics in a Marine Food Web. In Aquatic Food Webs: an Ecosystem Approach. Belgrano, A., Scharler, U., Dunne, J., and Ulanowicz, R. eds. Oxford University Press, pp. 19-24.\\
%\subsection{Research Papers: In second review after favorable reviews}
%\item {\bf Meli\'an, C. J.}, Alonso, D., V\'azquez, D. P., Regetz, J., and Allesina, S. (2009). Unifying Theories of Molecular, Community and Network Evolution. {\em Ecology Letters}\\
%\subsection{Research Papers: In revision and in preparation (last stage)}
%\item {\bf Meli\'an, C. J.}, Alonso, D., Allesina, S., Etienne, R. S., and Condit, R. (2009). A Biodiversity Number with Explicit Speciation.\\
%\item {\bf Meli\'an, C. J.}, Allesina, S., and Williams, R. J. (2009). Neutral Evolution Likelihood to Predict Multilevel Food Web Properties.\\
%\item {\bf Meli\'an, C. J.}, K\v{r}ivan, V., and Star\'y, P. (2009). Testing Metaweb Models at Local and Macroecological Scales.\\
%\end{enumerate}

\section{Invited Talks}\\
\\
{\bf 2010} Title: {\bf Exploring Ideas to Link Systems Biology and Ecosystem Dynamics}, ERDC, Environmental Laboratory, US Army Engineer Research and Development Center, March, Vicksburg, USA.\\
{\bf 2010} Title: {\bf Eco-Evo Dynamics in Networks: From Genomes to Radiations and Ecosystems}, Center for Ecology, Evolution and Biogeochemistry, EAWAG-University of Bern, February, Switzerland.\\
{\bf 2009} Title: {\bf Making connexions between systems biology and evolving ecological networks}, CoSBi-University of Parma, July, Italy.\\
{\bf 2009} Title (``keynote speaker''): {\bf Towards a general framework in food webs driven by data at multiple biological levels}, $2^{nd}$ International workshop ``Body size and ecosystem dynamics: Implications for conservation and management of natural resources'', ESF program ($2^{nd}$ SIZEMIC meeting), The Sven Lov\'en Center for Marine Sciences, University of Gothenburg, June, Sweden.\\
{\bf 2009} Title: {\bf How to form collaborations to answer big burning ecological questions?} {\em SEEDS Leadership Meeting}, Ecological Society of America, February, New Mexico, USA.\\
{\bf 2008} Title: {\bf Exploring Biodiversity Numbers for Multilevel Biological Networks}. Department of Ecology, Evolution and Organismal Biology, Iowa State University, November, Iowa, USA.\\
{\bf 2007} Title: {\bf Unifying Diversity Evolution in Multilevel Biological Networks}. Microsoft Research Ltd., November, Cambridge, UK.\\
{\bf 2007} Title: {\bf Evolution in Multilevel Complex Networks}. Santa Fe Institute, February, New Mexico, USA.\\
{\bf 2006} Title: {\bf Does Topology Alter Persistence in a Complex Mutualist/Antagonist Ecological Network?} Workshop on {\em Advances in Food Webs}, September, Yokohama, Japan.\\
{\bf 2003} Title: {\bf Effect of Habitat Loss on Complex Food Webs}. InterACT workshop on ``Community viability analysis: Identifying fragile systems and keystone species'', August, Link\"oping, Sweden.\\ 


\section{Invited Workshops}
\\
{\bf 2009} $1^{st}$ International Marine Interaction Webs Workshop, Sierra Nevada Research Institute, University of California-Merced, September, USA.\\
{\bf 2008} ``Trophic Dynamics in ecosystems: feeding interactions, species identity and body size'', $1^{st}$ SIZEMIC meeting ESF program, Clare College, University of Cambridge, April, UK.\\ 
{\bf 2007-2009} ``Towards a Unified Theory of Biodiversity'', NCEAS working group, Santa Barbara, California, USA.\\
{\bf 2004} ``The Effects of Species Loss at Higher Trophic Levels: Consequences for the Composition, Structure and Ecosystem Functioning of Food Webs'', (InterACT, European Science Foundation), September, Cork, Ireland.\\
{\bf 2004} ``From Structure to Dynamics in Complex Ecological Systems'', Santa Fe Institute, February, Santa Fe, New Mexico, USA.\\

%\vspace{0.2 in}
%\newpage
%{\Large \bf Research Visits and Scientific Semminars}
%
%{\bf 2002} Scripps Institution of Oceanography, San Diego, California, USA. (July-August and September).\\
%Visiting Prof. G. Sugihara. Non-random Assembly Models of Food Webs.\\
%{\bf 2003} Rockefeller University, New York, New York, USA. (June and July).\\
%Visiting Prof. J. E. Cohen. Assembling Abundance, Body Size and Interaction Strength in a Complex Food Web.\\
%{\bf 2007} Prof. Richard Condit. Bayesian Approaches to Community Ecology. NCEAS, USA. (April). 
\end{document}


 























 













