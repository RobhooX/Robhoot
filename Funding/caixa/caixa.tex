\documentclass[authoryear,1p,12pt]{elsarticle}
\usepackage{amsmath}
\usepackage{amssymb}
\usepackage[mathlines,displaymath]{lineno}
\usepackage{graphicx}
\usepackage[all,2cell,dvips]{xy}
\usepackage{natbib}
\usepackage[usenames,rgb]{xcolor}
\usepackage{ucs}
\usepackage[utf8]{inputenc}
\usepackage{array}
\usepackage{longtable}
\usepackage{calc}
\usepackage{multirow}
\usepackage{hhline}
\usepackage{ifthen}
% \usepackage{babel}
\usepackage{tcolorbox}
\newtcolorbox{mybox}{colback=grey!5!white,colframe=red!75!black}
\usepackage{setspace}
\usepackage{verbatim}
\usepackage{etaremune}
\usepackage{url}
% \usepackage{color}
\usepackage{graphicx}
\usepackage{natbib}
% \usepackage[latin1]{inputenc}
\usepackage{array}
\usepackage{float}
\usepackage{wrapfig}
\usepackage{hyperref}
\usepackage{ifpdf,eurosym,amstext}
%\floatstyle{boxed}
%\restylefloat{figure}

\newcommand{\etal}{{et~al.{}}}
\newcommand{\ie}{{i.~e.{}}}
\newcommand{\eg}{{e.~g.{}}}
\newcommand{\viz}{{viz.{}}}
\newcommand{\etc}{{etc.{}}}
\newcommand{\apriori}{{a priori{}}}
\newcommand{\vv}{{vice versa{}}}
\newcommand{\cf}{cf.{}}
\setcounter{tocdepth}{3}
\usepackage{tikz}
% \newcommand{\carlos}[1]{\textcolor{Red}{#1}}

% Text layout
\topmargin 0.0cm
\oddsidemargin 0.25cm
\evensidemargin 0.25cm
\textwidth 15.5cm
\textheight 21cm

\vspace{0.1 in}
\pagestyle{myheadings}
%\markboth{CJ Meli\'an \& VM Egu\'iluz}{CJ Meli\'an \& VM Egu\'iluz}


\begin{document}
\section*{{\bf Letter of Intent Caixa{\bf impulse}}}

\section{{\bf General Information}}

\subsection{{\bf Applicant Entity}}
Agencia Estatal Consejo Superior de Investigaciones Cient\'ificas CSIC

\subsection{{\bf Applicant Entity Tax ID Number ( NIF/CIF)}}
Q2818002D

\subsection{{\bf Project title (maximum characters: 100)}}
Automated Rapid Discovery and Transfer in Global Emergency Health Situations

\subsection{{\bf Project acronym}}
AUDAS

\subsection{{\bf Name of the project leader}}
V\'ictor Mart\'inez Egu\'iluz

\subsection{{\bf Select the business area}}
Digital Health

\subsection{{\bf Participation of the project in other ''la Caixa'' Foundation Calls}}
No

\subsection{{\bf What is the protection status of the Asset(s)?}}
No protection status

\subsection{{\bf Time to market}}
Less than 2 years

\section{{\bf Technology Readiness Level (TRL)}}
TRL 5 – Technology validated in relevant environment

\subsection{{\bf Has the Asset(s) been transferred to a Spin-off company?}}
No

\subsection{{\bf Amount requested from the "la Caixa'' Foundation}}

The total amount request will be of around 153k \euro{}. The following
is the list of the four main parts of the budget:
\begin{itemize}
\item {\bf Discover phase}:
  \item 2y Postdoctoral researcher: 78k \euro{}
  \item {\bf Transfer phase}:
  \item HealtHack workshop to develop reciprocal data-management and
    technology-transfer protocols at local level: 20k \euro{}
  \item Interface development from the experience gained in the
    HealthHack: 6m Data or Computer-scientist: 25k \euro{}
  \item International workshop with health managers to explore
    protocols for data-standards and scalability at the global scale:
    30k \euro{}
  \end{itemize}  

\subsection{{\bf Describe the core idea of your application in one sentence}}
Automated rapid discovery and transfer of data- and causal-knowledge
graphs to facilitate informed decision and complex problem solving in
human health ecosystems at local and global scales.

\subsection{{\bf Abstract}}
Global Health is a major goal of humanity challenged by several
pathogens along human history. Many studies have shown global health
and sustainability could be achieved by strengthening transparency and
feedbacks among social, ecological, technological and governance
systems. Global health and sustainability goals, however, strongly
depend on rapid access to evidence-, research-, discovery-, and
transfer-based knowledge gaps, especially in global emergency
situations. Yet, the science-technological-governance ecosystem lack
technologies narrowing down global knowledge gaps facilitating rapid
access to information to strengthen feedbacks among social,
ecological, technological and governance ecosystems. We propose a
rapid automated discovery and transfer technology targeting
knowledge-gaps and knowledge-causal graphs throughout reproducible and
rapid report generation in federated networks. A federated discovery
network encompasses a hybrid-automated-technology to lay out the
foundation of an open-, cooperative-science ecosystem to automate
discovery strengthening rapid solutions and robustness at the local
and global scale in emergency and global health challenges. The
project aims to provide the architecture of a science-enabled
technology to connect global human challenges to easy access
neutral-knowledge generation for rapid decisions in knowledge-inspired
societies.

\section{{\bf Scientific and technical feasibility}}

\subsection{{\bf What are the scientific fundamentals underlying your
    idea? Have you performed a proof of concept validation?}}
The scientific fundamentals of the present project are based on the
growing computing capacity, the integration between causal-knowledge
graphs and explainable Artificial Intelligence, and the interconection
of open-source technologies in the rapidly evolving digital
ecosystem. Targeting these fundamentals in automated discovery and
transfer requires the integration of different modules, from
identification and retrieval of heterogeneous data sources, to the
integration of explainable modeling and causal-inference, and
visualization and reporting. We have already advanced in the
integration of the different modules, from the automated
identification, retrieval and data integration to inference and
process-based discovery. We have implemented a prototype for the
ongoing covid-19 pandemic. Each module includes state-of-the-art
developments in computer science, complex systems, and theoretical
evolutionary ecology. The proof-of-concept is not fully automated yet
and still requires human intervention in module integration and the
development of a testnet stage. Nonetheless, we are currently
exploring innovative solutions especially in the modules of automated
data discovery, causal-knowledge graphs, reporting, and visualization.

\subsection{{\bf Describe how the Project’s Leader knowledge and
    experience in field of the Asset(s) would contribute to the
    success of the Project. Discuss it in the context of the
    composition and the relevancy of the rest of the team
    members.}}

The PL has a long tradition in interdisciplinary and computing related
projects, with team members coming from different fields. PL has
experience in health-related topics, in particular he has developed
collaborations with Harvard medical school and many biodiversity and
sustainability research institutions. The group of the PL has worked
in the development of data-driven agent-based networks in social,
biological and environmental problems with particular relevance in
epidemiological networks. Other team members with a broad expertise in
scientific computing and open-source development in the context of
biodiversity and global sustainability projects include Dr. Melian,
ETH-Domain, Switzerland.

\subsection{{\bf Has your Asset(s) been under evaluation by other funding
  programmes prior to this call} (an object of research grants or other
  innovation or acceleration programmes)? Please detail the most relevant ones}
No

\section{{\bf Transferability and market potential}}

\subsection{According to the stage of development of the Asset(s), what
  are {\bf the identified needs and determinants that currently condition
  its successful progression} to the commercialisation phases?}}

During the development process of the prototype (i.e., Robhack
workshop during March 2020 and a git repository with five-to-ten
contributors per day), we have identified the following needs
currently conditioning the progression of the asset.

\begin{itemize}
\item From the human resources perspective, team members have
  volunteered in a coordinated way using collaborative open-source
  software to develop and integrate the different modules of the
  asset. We are currently in need of two researchers, a postdoctoral
  with expertise in Julia computing language and a computer-scientit
  or developer for interface implementation. The team has recently
  participated in a intensive one-week Robhack event, i.e., hackaton
  style for automated technologies, where the different modules have
  been identified, integrated and tested.
\item From the technical data-integration part, while we have been
  able to integrate state-of-the-art algorithms, there are still some
  gaps that need to be addressed. For example, data discovery for
  biomedical knowledge bases requires highly general and
  fault-tolerance algorithms to access heterogeneous APIs. We were
  successful in reaching many APIs types for Covid-19 data, but still,
  more general and fault-tolerance algorithms are required to automate
  API discovery and data-knowledge graphs.
\item From the technical AI and causal-knowledge graph part, we have
  been able to test a variety of fitting methods to large and complex
  causal-knowledge graphs to obtain the sequence of events that best
  fit to the demography parameters in the Covid-19 at city and global
  scales (i.e., susceptibles, incubation period, infected, recovered,
  and deaths). Yet, generalizing neural nets and causal-knowledge
  graphs will require novel implementations of neural ordinary
  differential methods to rapidly optimize models of thousands of
  parameters.
\end{itemize}

\subsection{{\bf Describe the specific innovation transfer
  milestones you wish to achieve} by participating in this Call in
  terms of time-to-market}

The participation in this call will allow us to transfer to the local
and national health system an automated discovery platform using
locally collected and standarized data to show the importance of rapid
and reproducible discovery for predicting complex problems at the
interface of social and governance systems. Specifically, to reach the
market, the following milestones have been identified: At the end of
the first year a full operation Asset will be available and tested in
a human health case study: the propagation of the covid pandemic at
local and global scales (2y Postdoctoral resercher). During the second
year, the Asset will be finished with a user-friendly interface
environment aiming to be useful for the non-expert (6m Computer
scientist or developer).
  
  \subsection{{\bf Who are your end-users?} What is the market segment
    and the market size you are addressing?}

  We identified as potential end-users of our Asset, a person,
  organization or governance entity that generate
  research-based-knowledge and/or requires reports for decision
  making. Specifically:

  \begin{itemize}
  \item Health Labs and biomedicine institutions: While part of
    the research is used to analyze data and modeling, a large part
    of the community are end-users that use packages or software to
    either analyze data and/or modeling. This sector will benefit from
    the Asset as it will help to obtain automated robust results.
  \item Public administration: Decision making and governance require
    analysis of the risks and emergency situations at local and global
    scales to make decisions that will be beneficial collectively at
    these scales. For example, the ongoing pandemic is showing us the
    need to account for almost-real-time automated analysis of the
    situation where most of the epidemiological variables are
    unknown. The Asset will assist decision-makers in this regard by
    providing rapid and reproducible reporting accounting for the
    state-of-the-art data and modeling technologies.
  \item Private companies and corporations: The growth and survival of
    a company often requires constant innovation and evolution. The
    Asset will facilitate the discovery cycle in private companies
    offering them rapid updates of new emerging situations, localy and
    globally.
\end{itemize}

\section{{\bf Societal relevance and potential impact}}


\subsection{{\bf What is the unmet need you aim to address} and why is it
    timely and relevant?}

  Access to information is key for modern societies facing complex
  problems including global health. Living in an ultra-connected world
  where news travels constantly and fast from one point to the other
  around the globe might produce misinformation and information
  deluge. The growth of the so-called fake news makes the picture even
  more complex. This is particularly timely and relevant for
  anticipating decisions on unexpected and complex health emergency
  and sustainability situations at local, regional and global
  scales. To solve these problems, private companies emerge for
  example as validators of the information. Dentralization and
  transparency from the source of the data to the reporting is key to
  facilitate the validation of the information used to solve complex
  problems. Our asset will contribute to facilitate reproducible
  reporting generation by providing a the tools and the interface at
  the local, individual level.

\subsection{{\bf What potential impact will your solution have?} (maximum
  characters: 1500)}
Automated discovery can have a large impact to people in need to
access rapid, robust, and reproducible reportings to take informed
decisions, especially in global emergency and/or in sustainability
situations.

\subsection{{\bf What are the available solutions that currently address
  the specific problem you aim to solve?}}

Most of the available solutions offering rapid and semi-automated
reporting generation to solve complex problems in health emergency
situations at local and global scales are based in highly fragmented,
non-reproducible and black-box technologies. This situation currently
inhibits the rapid access of many local companies and governance
entities to a technology offering integration, reproducibility and
transparency of the reportings to deal with rapid uncertain coonditions
in health emergency situations.

\subsection{{\bf What is the innovation of your solution? Who will be the
  main beneficiaries of your solution? (maximum characters: 1500)}}
Our innovation to help take informed decisions in emergency and
sustainability challenges come from two fronts: First, we will
integrate highly fragmented and non-reproducible technologies into a
compact reproducible and automated technology. Second, novelty comes
from automating data-, and causal-knowledge graphs into an integrated
workflow. We aim to release a completely free and open-source
prototype for developers to facilitate robustness and scalability of
the technology. We will have additional features to offer to research
labs and institutions, public administration and private bussiness to
facilitate explainable and reproducible reporting generation.

\section{{\bf Additional Documentation}}
Copy of the identification document of the Project Leader
(DNI/NIF/ID)Letter of commitment from the host institution (template
here)


\end{document}


\section{References}
\bibliographystyle{unsrtnat}
\bibliography{ref.bib}
\end{document}
